\chapter{CLAW Primitives}
\label{chapter:primitives}
CLAW transformations, low-level or high-level, are built on various smaller 
blocks that can be re-used in other transformations. Those blocks are called
primitives as they define a well constraint test or transformation on the
\xcodeml AST. These primitives block use the \xcodeml AST abstraction library
detailed in Chapter \ref{chapter:astmanip}.
The different primitives available in the CLAW framework are listed in this
chapter. 
%TODO complete the chapter with all primitives
\textit{!!! CURRENTLY NOT COMPLETE !!!}.

\section{Loop primitive}
In this section, primitive transformations applicable on do statements are 
detailed.

\subsection{Loop fusion (merge)}
Loop fusion or merge is the process of merge the body of a slave loop into
the one of a master loop.
In \xcodeml, a \lstinline|FdoStatement| has a child \lstinline|body| node.
This \lstinline|body| node has 0 to N children. Performing a merge of two
\lstinline|FdoStatement| is done by shifting all the element of the slave
\lstinline|FdoStatement->body| to the master \lstinline|FdoStatement->body|.

\subsubsection{API}
Signatures of the available methods to perform a loop fusion are shown in
Listing \ref{lst:pri_merge}. The first is used to perform a simple fusion 
on two simple \lstinline|FdoStatement|. The second uses 
\lstinline|NestedDoStatement| objects and will perform the merge of the 
inner most bodies.

\begin{lstlisting}[label=lst:pri_merge, language=java,title=LoopTransform.java]
public static void merge(Xnode master, Xnode slave);
public static void merge(NestedDoStatement master, NestedDoStatement slave);
\end{lstlisting}

\subsubsection{Implementation}
The implementation is quite simple.
\begin{itemize}
  \item Check the two nodes are not null and actual \lstinline|FdoStatement|
  \item Call the \lstinline|Body.appendBody| method that append the body of the
        second node to the first node body.
  \item Clean-up any \lstinline|FpragmaStatement| that were decorating the 
        slave \lstinline|FdoStatement|
  \item Delete the slave \lstinline|FdoStatement| as it is now included in the
        master one.
\end{itemize}

\begin{lstlisting}[label=lst:loop_merge, language=java, title=Loop.java]
public static void merge(Xnode masterDoStmt, Xnode slaveDoStmt)
      throws IllegalTransformationException
{
  if(masterDoStmt == null || slaveDoStmt == null
      || masterDoStmt.opcode() != Xcode.FDOSTATEMENT
      || slaveDoStmt.opcode() != Xcode.FDOSTATEMENT)
  {
    throw new IllegalTransformationException(
        "Incompatible node to perform a merge");
  }

  // Merge slave body into the master body
  Body.append(masterDoStmt.body(), slaveDoStmt.body());

  // Delete any acc loop / omp do pragma before/after the slave do statement.
  Loop.cleanPragmas(slaveDoStmt, prevToDelete, nextToDelete);
  
  // Delete the slave do statement
  slaveDoStmt.delete();
}
\end{lstlisting}


\begin{lstlisting}[label=lst:body_append, language=java, title=Body.java]
private static void appendBody(Xnode masterBody, Xnode slaveBody)
    throws IllegalTransformationException
{
  if(masterBody == null || slaveBody == null
      || masterBody.opcode() != Xcode.BODY
      || slaveBody.opcode() != Xcode.BODY)
  {
    throw new IllegalTransformationException(ClawConstant.ERROR_INCOMPATIBLE);
  }

  // Append content of slave body master body
  Xnode crtNode = slaveBody.firstChild();
  while(crtNode != null) {
    Xnode nextSibling = crtNode.nextSibling();
    masterBody.append(crtNode);
    crtNode = nextSibling;
  }
}
\end{lstlisting}

\subsection{Loop reorder}
%TODO

\section{Field primitive}
%TODO


\subsection{Scalar and arrays promotion}
Promotion of a scalar or array field is defined as a primitive transformation.
\lstinline|FieldTransform.promote| is the only method to be called to perform
this transformation on a field.

A promotion is defined by various elements. First, the dimension definition 
describes a dimension that will be used for the promotion.
In a promotion process, 1 to N dimension are added to an existing field type. 
Each dimension definition can describe its insertion position regarding the 
existing dimensions.


\begin{lstlisting}[language=fortran]
REAL, DIMENSION(1:10,1:20) :: a

! Some dimensions used to illustrate the promotion process
! Dimension description ndim(1:30)
! Dimension description zdim(1:30)

! Promoted with ndim before existing dimension
REAL, DIMENSION(ndim,1:10,1:20) :: a
! Promoted with ndim in middle existing dimensions
REAL, DIMENSION(1:10,ndim,1:20) :: a
! Promoted with ndim after existing dimensions
REAL, DIMENSION(1:10,1:20,ndim) :: a

! Promoted with ndim and zdim before existing dimension
REAL, DIMENSION(ndim,zdim,1:10,1:20) :: a
! Promoted with ndim and zdim in middle existing dimensions
REAL, DIMENSION(1:10,ndim,zdim,1:20) :: a
! Promoted with ndim,zdim after existing dimensions
REAL, DIMENSION(1:10,1:20,ndim,zdim) :: a

! Promoted with ndim before and zdim after existing dimension
REAL, DIMENSION(ndim,1:10,1:20,zdim) :: a
! Promoted with ndim in middle and zdim after existing dimensions
REAL, DIMENSION(1:10,ndim,1:20,zdim) :: a
! Promoted with ndim before and zdim in middle existing dimensions
REAL, DIMENSION(1:10,1:20,ndim,zdim) :: a
\end{lstlisting}
