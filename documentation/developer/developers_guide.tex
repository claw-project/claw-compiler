\documentclass[a4paper, 11pt]{report}
\author{Valentin Clement}
\title{CLAW Fortran Compiler - Developer's Guide}

\usepackage[utf8]{inputenc}
\usepackage{verbatim}
\usepackage{moreverb}
\usepackage[english]{babel}
\usepackage[T1]{fontenc}
\usepackage{lmodern}
\usepackage{graphicx}
\usepackage{fancyhdr}
\usepackage{listings}
\usepackage{lastpage}
\usepackage[top=2cm, bottom=1cm, left=2cm, right=2cm]{geometry}
\usepackage{color}
\usepackage[table]{xcolor}
\usepackage{graphicx}
\usepackage{lipsum}
\usepackage{pgfplots}
\usepackage{CJKutf8}
\usepackage{glossaries}

\setcounter{secnumdepth}{3}


\def\clawfc{CLAW FORTRAN Compiler}
\def\omni{OMNI Compiler Infrastructure}

\makeglossaries

\newacronym{ir}{IR}{Intermediate Representation}


%font change
\renewcommand{\familydefault}{\sfdefault}

\newcommand{\HRule}{\rule{\linewidth}{0.5mm}}
\newcommand{\nl}{\\[0.1cm]}
\newcommand{\s}{\vspace{0.3cm}}
%\newcommand{\emptypage}{\newpage \thispagestyle{empty} \mbox{}\newpage}
\newcommand{\emptypage}{}
\newcommand{\smore}{\vspace{0.6cm}}

\usepackage{caption}
\DeclareCaptionFont{white}{\color{white}}
\DeclareCaptionFormat{listing}{\colorbox{gray}{\parbox{\textwidth}{#1#2#3}}}
\captionsetup[lstlisting]{format=listing,labelfont=white,textfont=white}

\definecolor{darkgreen}{rgb}{0,0.4,0}
\definecolor{mauve}{rgb}{0.58,0,0.82}
\definecolor{Gray}{rgb}{0,0,0}
\definecolor{LightGray}{gray}{0.9}


\lstset{
    basicstyle=\footnotesize\ttfamily,
    keywordstyle=\color{orange}, %keywordstyle=\color{MidnightBlue}\bfseries,
    identifierstyle=\color{black},
    commentstyle=\color{darkgreen},
    stringstyle=\color{red},
    numbers=left,
    numberstyle=\color{Gray}\tiny,
    frame=bt, %frame=single,
    rulecolor=\color{Gray},
    numbersep=7pt,
    extendedchars=true,
    captionpos=b,
    breaklines=true,
    showspaces=false,
    showtabs=false,
    tabsize=2,
    xleftmargin=20pt,
    framexleftmargin=20pt,
    framexrightmargin=0pt,
    framextopmargin=0pt,
    framexbottommargin=0pt,
    showstringspaces=false,
    aboveskip=20pt,
    belowskip=20pt
}

\lstset{
	emph={parclass, sync, async, broadcast, scatter, gather, reduce, seq, conc, mutex},
	emphstyle={\color{orange}\bfseries}
}

\addto\captionsenglish{%
  \renewcommand{\listfigurename}{List of figures}%
	\renewcommand\refname{}
}

%Header and footer
\pagestyle{fancy}
\fancyhead{}
\fancyfoot{}

%Header definition
\renewcommand{\headrulewidth}{0.5pt}
\lhead{CLAW FORTRAN Compiler}
\rhead{Developer's Guide}

%Footer definition
\renewcommand{\footrulewidth}{0.5pt}
\lfoot{Version 0.1}
\rfoot{Page \thepage ~on \pageref{LastPage}}

\definecolor{grey}{rgb}{0.9,0.9,0.9} % Color of the box surrounding the title - these values can be changed to give the box a different color	

%remove indent for paragraph
\parindent0ex

\begin{document}

\thispagestyle{empty} % Remove page numbering on this page

%----------------------------------------------------------------------------------------
%	TITLE SECTION
%----------------------------------------------------------------------------------------
\colorbox{grey}{
	\parbox[t]{1.0\linewidth}{
		\centering \fontsize{35pt}{80pt}\selectfont % The first argument for fontsize is the font size of the text and the second is the line spacing - you may need to play with these for your particular title
		\vspace*{2cm} % Space between the start of the title and the top of the grey box
		
		\hfill \textbf{CLAW FORTRAN Compiler} \\
		\hfill Developer's Guide\par
		
		\vspace*{2cm} % Space between the end of the title and the bottom of the grey box
	}
}

\vfill

\begin{center}
\includegraphics[width=4cm]{resources/c2sm_logo.pdf} \\
\end{center}

\vfill % Space between the title box and author information

\begin{center}
Version 0.1, Last updated \today
\end{center}
\HRule

\clearpage % Whitespace to the end of the page



\newgeometry{top=3cm,bottom=3cm,right=2cm,left=2cm}

%\emptypage
%\pagebreak
%\label{chapter:ack}
%\begin{center}
%\textsc{\LARGE Acknowledgment}
%\end{center}
%\input{acknowledgment}

%\emptypage
%\pagebreak
%\label{chapter:abstract}
%\begin{center}
%\textsc{\LARGE Abstract}
%\end{center}


\emptypage
\pagebreak
\tableofcontents



\chapter{Architecture}
\section{OMNI Compiler}
The \clawfc is based on the \omni. 

\section{XcodeML/F}
XcodeML/F is the \gls{ir} used by the translator in the CLAW FORTRAN Compiler. This \gls{ir} is based on the XML format and is described in a specification document. It allows to have an high-level representation of FORTRAN 2008 programs.
%TODO add reference to specs.

Listing \ref{fortran1} is a simple FORTRAN program. Its XcodeML/F \gls{ir} will be as shown in Listing \ref{xcodeml1}. A typical XcodeML/F translation unit is composed of the followings sections: 
\begin{itemize}
\item Type table with all the type definitions used in the translation unit (Listing \ref{xcodeml1} line 6-24).
\item Global symbol table listing all the symbol at global scope (Listing \ref{xcodeml1} line 25-29).
\item Global declaration section listing the actual function or module declarations (Listing \ref{xcodeml1} line 30-82).
\end{itemize}

\lstinputlisting[label=fortran1, caption=Basic FORTRAN program, language=Fortran]{code/basic_fortran.f90}

\lstinputlisting[label=xcodeml1, caption=XcodeML/F IR, language=xml]{code/basic_fortran.xml}

\section{CLAW XcodeML to XcodeML translator}


\chapter{Transformation}
\section{Transformation application order}

\section{Add a transformation}
A transformation is always triggered by a directive. 

% GLOSSARY
\pagebreak
\glsaddall
\printglossaries

\emptypage
% FIGURES
\pagebreak
\listoffigures

%TABLES
\pagebreak
\listoftables

\emptypage
% LISTINGS
\pagebreak
\lstlistoflistings

% REFERENCES
%\input{references}

%\emptypage
%\input{appendix}

\end{document}
